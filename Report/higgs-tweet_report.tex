\documentclass[sigconf]{acmart}

\usepackage{geometry}                		
\geometry{letterpaper}                   		                		
\usepackage[parfill]{parskip}    		
\usepackage{graphicx}
\usepackage{subfig}
\usepackage{float}
\usepackage{color}
\usepackage{amssymb}
\usepackage{amsfonts}
\usepackage{amsmath}

\usepackage[framed, useliterate]{mcode}
\usepackage{listings} % For displaying code
\usepackage{algorithm}
\usepackage{algorithmic}
\usepackage{hyperref}

\acmISBN{}
\acmDOI{} 
\setcopyright{none}
\settopmatter{printacmref=false, printccs=false, printfolios=true}

\title{HiggsTweet: Analyzing Influence Propagation During a Viral Event on Twitter}

\author{Kristian Flatheim Jensen}
\affiliation{%
\institution{Norwegian University of Science and Technology}
}

\author{Gudbrand Tandberg}
\affiliation{%
\institution{The University of British Colombia}
}

\begin{document}
\maketitle

\begin{abstract}
In this project we analyze the influence dynamics within a certain interest-group during a viral event. The event in question is the 4. July 2012 discovery of the Higgs boson, made by researchers at CERN in Switzerland. We study a dataset consisting of a 450k-user Twitter follower network together with a 13M line event log, collected between the 1st and 7th of July 2012. We use the event log to calculate influence probabilities between pairs of users, these weights are then used to run simulations of Independent Cascade (IC) diffusion processes and compute near-optimal seed sets using a greedy algorithm. We experiment with several different preprocessing steps and heuristics for reducing the size and runtime of the simulation and optimization steps, and compare seed-sets and expected user influence across our experiments. 
\end{abstract}

\section{Introduction}

Analyzing the spread of information through a large and complex social network is a difficult and interesting problem. In the last 10 years, the Big Data revolution has been driven, to a large extent, by innovations in understanding how humans interact and live in a mobile world. At least some of this progress has been made thanks to inter-disciplinary efforts by researchers in sociology, epidemology, psychographics, computer science. Already it has become clear the immense opportunities and dramatic shifts this revolution has brought about for marketers, news agencies, individuals and more. We will see in the near dystopian future that the study of social networks will in fact be key to developing political theory (and practice) into the 21st century.

The main goal of the broader research agenda we are following is the open-ended and general question of analyzing the power and influence dynamics of a web-community before, during, and after a major event. This study of course has a much narrower scope. Our lower level goals for this project were twofold, first, we wanted to get hands-on experience with some of the material we covered in class, notably Influence Maximization (IM), second, we wanted to perform an as-complete-as-possible scientific exposition of a new and interesting dataset. We will see how successful we were at the end!

Some questions that guided our efforts are 
\begin{itemize}
\item \emph{Which users in the network should we influence, and when should we influence, if we wanted to spread a rumor during a viral event?}

\item \emph{How do the power dynamics, as computed from an event log change over time during a viral event?}

\item \emph{What does the typical and the atypical user look like, in terms of event history, during a viral event?}

\item \emph{Do the seed-sets (as computed using IM) differ significantly from time to time, or is there overlap?}

\item \emph{Can the selection of seed sets be simplified, or substituted for other statistics- or feature-based heuristics? }

\end{itemize}

\section{Preliminaries \& Related Work}

Network diffusion processes have a long history of study in the social sciences. Some of the early applications include modeling the adoption of new products and technologies, modeling disease outbreaks and word-of-mouth events, and understanding human social dynamics. With the relatively recent advent of global social network platforms such as Facebook and Twitter, many new lines of research in computer science have emerged. Particularly, the processes of influence propagation in social networks has received a lot of attention.

In the age of Youtube, Facebook and Twitter, the appeal of viral marketing is to many the ultimate free lunch: Pick some small number of people to "seed" your idea, get it to "go viral", and watch while it relentlessly spreads to reach millions, all on a shoestring budget. In \cite{watts2007viral}, the authors instead propose a new model called "Big Seed Marketing" that combines the power of traditional advertising and the extra punch provided by viral propagation. This follows from years on marketing research, see for example \cite{domingos2001mining} for an early study in computing the "value" of a user in a social network. 

The problem of selecting a set of seed-users that will trigger a large cascade of activity was first introduced in \cite{kempe2003maximizing}. In this seminal work the computational problem of maximizing the expected spread of a seed set is formulated as a discrete optimization problem, proven to be NP-hard, and a greedy algorithm with provable approximation guarantees is presented for a class of diffusion models. Importantly, they find that they are able to attain significantly higher values of expected spread by solving the Influence Maximization (IM) problem, as opposed to both random and centrality-based seed-selection methods. Since the greedy algorithm can be relatively slow to run on large networks, several methods have been presented to deal more efficiently with the IM problem. These include the optimized algorithms CELF, MIA, TIM and IMM, \cite{chen2009efficient}, \cite{chen2010scalable}, \cite{tang2014influence}, \cite{tang2015influence}. 

The methods for solving the IM problem all depend on a directed, weighted social network as input. The weight of a directed edge between two users is supposed to represent the degree of influence the one has over the other. Estimating this number is an interesting and general question in itself, and there is still plenty of room for more research on it. The problem of estimating influence probabilities was first presented in \cite{goyal2010learning}, where the authors present static and time-dependent models for learning influence weights from a log of events. 

The related question of identifying influential spreaders in complex networks was partly answered in \cite{kitsak2010identification}. In it, they find that there are circumstances where the best spreaders do not correspond to the best connected people or to the most central people (high betweenness centrality), rather, they are located within the core of the network as identified by the k-shell decomposition analysis, and that when multiple spreaders are considered simultaneously, the distance between them becomes the crucial parameter that determines the extent of the spreading. More answers to the same question came in \cite{bakshy2011everyone}, where the authors investigate the attributes and relative influence of a large Twitter follower graph over a two month interval in 2009. They find that the largest cascades tend to be generated by users who have been influential in the past and who have a large number of followers. They also find that hashtags that were rated more interesting and/or elicited more positive feelings were more likely to spread. They also find that predictions of which particular user will generate large cascades are relatively unreliable. For an excellent survey of computational models of influence propagation, see e.g. \cite{bonchi2011influence}, or \cite{gomez2012inferring}.

In \cite{wang2010community}, the authors present a community-based algorithm for mining top-k influential nodes in social networks. Their method is found to lead to a decrease in runtime, while not sacrificing much in terms of spread. This is perhaps not so surprising, given the above result that distance between seed-users becomes the crucial parameter that determines the extent of the spreading. 

Combining the problems of influence maximization, influence estimation, and viral event dynamics, we are faced with the problem of real-time IM on dynamic social networks. This has been addressed in \cite{rodriguez2012influence}, and more recently in \cite{wang2017real}.

Some other relevant and recent research that we have taken inspiration from include studies of differences in mechanics of diffusion across topics \cite{romero2011differences}, the dynamics of protest recruitment \cite{gonzalez2011dynamics}, sentiment reciprocality in reply networks \cite{bliss2012twitter} and prediction of social-link creation times \cite{meeder2011we}.

The Higgs dataset was first presented in \cite{de2013anatomy}, where the authors present the dataset, explore the spatio-temporal properties of the data, and demonstrate a model for the information spreading in the social network during the event.

\section{The Data Set}




\section{Our Approach}

Combining social network $+$ "action edges" and running community based IM. 

\subsection{Computing Edge Probabilities}

\textbf{Idea:} Use WC probs with different weights for different action-types. 

\begin{itemize}

\item Hard to validate

\item Hard to compare

\item Effects of $p_{uv}$ on runtime. 

\item Effects of $p_{uv}$ on spread. 

\end{itemize}


\subsection{Influence Maximization}

Greedy, CELF, or MIA? What to do..

\textbf{Idea:} Divide and conquer--preprocess with community detection. Wang (2012)

\section{Results} 

\section{Discussion}

\subsection{Future Work}

Impact of time. 

Content of tweets - sentiment analysis. 

Compute $p_{uv}$ using unsupervised learning approach. Perform evaluative analysis. 

\nocite{*}
\bibliography{bibliography.bib}
\bibliographystyle{plain}

\end{document} 
