\documentclass[sigconf]{acmart}

\usepackage{geometry}                		
\geometry{letterpaper}                   		                		
\usepackage[parfill]{parskip}    		
\usepackage{graphicx}
\usepackage{subfig}
\usepackage{float}
\usepackage{color}
\usepackage{amssymb}
\usepackage{amsfonts}
\usepackage{amsmath}

\usepackage[framed, useliterate]{mcode}
\usepackage{listings} % For displaying code
\usepackage{algorithm}
\usepackage{algorithmic}
\usepackage{hyperref}

\acmISBN{}
\acmDOI{} 
\setcopyright{none}
\settopmatter{printacmref=false, printccs=false, printfolios=true}

\title{HiggsTweet: Analyzing Influence Propagation During a Viral Event on Twitter}

\author{Kristian Flatheim Jensen}
\affiliation{%
\institution{Norwegian University of Science and Technology}
}

\author{Gudbrand Tandberg}
\affiliation{%
\institution{University of British Colombia}
}

\begin{document}
\maketitle

\begin{abstract}
This is the abstract
\end{abstract}

\section{Introduction}

Analyze the power dynamics in the physics web-community at the time of a viral event.

Q: Which users to influence if we wish to spread scientific misinformation? 

\section{Related Work}

\nocite{*}

\section{Preliminaries}

Present IM problem

\section{The Data Set}

Social Network + Action Log

500k users; 14M edges (following/followee); 500k directed, typed actions

\section{Our Approach}

Combining social network $+$ "action edges" and running community based IM. 

\subsection{Computing Edge Probabilities}

\textbf{Idea:} Use WC probs with different weights for different action-types. 

\begin{itemize}

\item Hard to validate

\item Hard to compare

\item Effects of $p_{uv}$ on runtime. 

\item Effects of $p_{uv}$ on spread. 

\end{itemize}


\subsection{Influence Maximization}

Greedy, CELF, or MIA? What to do..

\textbf{Idea:} Divide and conquer--preprocess with community detection. Wang (2012)

\section{Results} 

\section{Discussion}

\subsection{Future Work}

Impact of time. 

Content of tweets - sentiment analysis. 

Compute $p_{uv}$ using unsupervised learning approach. Perform evaluative analysis. 

\bibliography{bibliography.bib}
\bibliographystyle{plain}

\end{document} 
